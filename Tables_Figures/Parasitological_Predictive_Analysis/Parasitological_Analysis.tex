\documentclass{beamer}
\usepackage[utf8]{inputenc}
\usepackage{booktabs}   % for cleaner table lines
\usepackage{longtable}  % if table spans multiple pages

\usepackage[table,xcdraw]{xcolor}  % Enables \cellcolor with gray!10
%\usepackage{colortbl}              % Optional, but good for full table coloring

\usetheme{Madrid}
\usecolortheme{default}
\usepackage{outlines}
%------------------------------------------------------------
%This block of code defines the information to appear in the
%Title page
\title[Predictive Analysis] %optional
{Parasitological Predictive Analysis}

\subtitle{NSF-Senegal}
\author{Kateri Mouawad} % (optional)

% {A.~B.~Arthur\inst{1} \and J.~Doe\inst{2}}

% \institute[VFU] % (optional)
% {
%   \inst{1}%
%   Faculty of Physics\\
%   Very Famous University
%   \and
%   \inst{2}%
%   Faculty of Chemistry\\
%   Very Famous University
% }

\date[June 2025] % (optional)
{NSF-Seneal Project Team Meeting}

%\logo{\includegraphics[height=1cm]{overleaf-logo}}

%End of title page configuration block
%------------------------------------------------------------

\begin{document}

\frame{\titlepage}

%The next statement creates the title page.
\begin{frame}
\frametitle{Analysis Overview}
This analysis proceeds in two steps for the joint-infection sm\_sh\_inf indicator:

\begin{itemize}
    \item<1-> Lasso-Logit Regression
       \begin{itemize}
             \item<2-> identifies the optimal lambda
             \item<3-> fits the model
             \item<4-> extracts the non-zero (important) variables.
             \item<5-> identifies most predictive covariates for sm\_sh\_inf
        \end{itemize}
    \item<3-> Evaluation using Confusion Matrices
         \begin{itemize}
                 \item<4-> We apply the fitted model to the test data and generate confusion matrices to assess predictive performance
         \end{itemize}
\end{itemize}

\end{frame}


\begin{frame}
\frametitle{Lasso-Logit Model Specification}

Best predictors that were selected by Lasso:


\end{frame}

\begin{frame}
\frametitle{Confusion Matrix}
The following output are the FPRs and FNRs using the logit-lasso coefficients. These are summarized in one confusion matrix.

\begin{table}[!h]
\centering
\caption{Confusion Matrix for sm\_sh\_inf at each threshold}
\centering
\begin{tabular}[t]{rrrrrrrr}
\toprule
Threshold & TP & TN & FP & FN & FPR & FNR & Sum\_FPR\_FNR\\
\midrule
\cellcolor{gray!10}{0.1} & \cellcolor{gray!10}{65} & \cellcolor{gray!10}{1} & \cellcolor{gray!10}{38} & \cellcolor{gray!10}{0} & \cellcolor{gray!10}{0.974} & \cellcolor{gray!10}{0.000} & \cellcolor{gray!10}{0.974}\\
0.2 & 64 & 2 & 37 & 1 & 0.949 & 0.015 & 0.964\\
\cellcolor{gray!10}{0.3} & \cellcolor{gray!10}{64} & \cellcolor{gray!10}{2} & \cellcolor{gray!10}{37} & \cellcolor{gray!10}{1} & \cellcolor{gray!10}{0.949} & \cellcolor{gray!10}{0.015} & \cellcolor{gray!10}{0.964}\\
0.4 & 62 & 7 & 32 & 3 & 0.821 & 0.046 & 0.867\\
\cellcolor{gray!10}{0.5} & \cellcolor{gray!10}{51} & \cellcolor{gray!10}{18} & \cellcolor{gray!10}{21} & \cellcolor{gray!10}{14} & \cellcolor{gray!10}{0.538} & \cellcolor{gray!10}{0.215} & \cellcolor{gray!10}{0.754}\\
\addlinespace
0.6 & 34 & 30 & 9 & 31 & 0.231 & 0.477 & 0.708\\
\cellcolor{gray!10}{0.7} & \cellcolor{gray!10}{25} & \cellcolor{gray!10}{37} & \cellcolor{gray!10}{2} & \cellcolor{gray!10}{40} & \cellcolor{gray!10}{0.051} & \cellcolor{gray!10}{0.615} & \cellcolor{gray!10}{0.667}\\
0.8 & 8 & 38 & 1 & 57 & 0.026 & 0.877 & 0.903\\
\cellcolor{gray!10}{0.9} & \cellcolor{gray!10}{4} & \cellcolor{gray!10}{38} & \cellcolor{gray!10}{1} & \cellcolor{gray!10}{61} & \cellcolor{gray!10}{0.026} & \cellcolor{gray!10}{0.938} & \cellcolor{gray!10}{0.964}\\
\bottomrule
\end{tabular}
\end{table}


%include a graph too for spice?


\end{frame}
%---------------------------------------------------------

\end{document}